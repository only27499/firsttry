\documentclass[a4paper,12pt,twoside]{ctexart}

\usepackage{amsmath, amssymb, amsfonts}% 数学符号
\usepackage{graphicx,float}%图与浮动环境
\usepackage{enumerate}% 自定义标号(不加就用这个 \item[label] description)
\usepackage{cite}% 引用\cite{}
\usepackage{booktabs}% 提供三线表的 \toprule,\midrule,\bottomrule
\usepackage{fancyhdr}% 页眉页脚 fancy 样式宏包,具体在各部分设置
%\usepackage[nottoc]{tocbibind}

\usepackage{subcaption}
\usepackage{enumerate}
\usepackage{xparse}% 用于重新定义命令



\usepackage{chngcntr}%重置每一章公式编号





%书签(同时设置无超链接无红色边框,此法更简洁)
\usepackage[bookmarks=true, colorlinks, citecolor=black, linkcolor=black]{hyperref}
%设置超链接无红色边框
%\usepackage{hyperref}
%\hypersetup{hidelinks,
%	colorlinks=true,
%	allcolors=black,
%	pdfstartview=Fit,
%	breaklinks=true
%}

%添加 pdf 文档属性
\hypersetup{
	pdftitle={“冯如杯” LaTeX 模板},
	pdfauthor={Martin Wilson a.k.a Shijia Cong},
	pdfcreator={Martin Wilson a.k.a Shijia Cong},
	pdfproducer={TexStudio with XeLaTeX, in TeXLive}
}


%页边距
\usepackage{geometry}
\geometry{a4paper,left=30mm,right=20mm,top=25mm,bottom=25mm}


%中文字体设置
%华文中宋定义
\setCJKfamilyfont{huawenzhongsong}{STZhongsong}[AutoFakeBold]%允许伪粗体
\newcommand{\huawenzhongsong}{\CJKfamily{huawenzhongsong}}
%华文新魏定义
\setCJKfamilyfont{huawenxinwei}{STXinwei}
\newcommand{\huawenxinwei}{\CJKfamily{huawenxinwei}}



%西文字体设置
\usepackage[T1]{fontenc}
\usepackage{mathptmx}



%标题设置
\usepackage{titlesec}
%\titleformat{章节命令}[形状]{格式}{标题序号}{序号与标题间距}{标题前命令}[标题后命令]
%\titleformat{\chapter}[block]{\LARGE \bfseries}{Chapter \arabic{chapter}}{1em}{}[]
%\titlespacing*{章节命令}{左边距}{上文距}{下文距}[右边距]
%\titlespacing{\chapter}{0pt}{-2pt}{2pt}

% 标题与副标题
\newcommand{\biaoti}{\zihao{2} \huawenzhongsong}%标题,华文中宋,二号,加粗
\newcommand{\fubiaoti}{\hfill \zihao{3} \huawenxinwei}%副标题,华文新魏,三号
% 章标题
\titleformat{\section}[block]{\heiti\centering\zihao{3}\bfseries}{第\chinese{section}章}{1em}{}[]
\titlespacing{\section}{0pt}{0.5ex}{0.5ex}
% 节标题
\titleformat{\subsection}[block]{\heiti\raggedright\zihao{4}\bfseries}{\arabic{section}.\arabic{subsection}}{1em}{}[]
\titlespacing{\subsection}{0pt}{0.5ex}{0.5ex}
% 条标题
\titleformat{\subsubsection}[block]{\heiti\raggedright\zihao{-4}\bfseries}{\arabic{section}.\arabic{subsection}.\arabic{subsubsection}}{1em}{}[]
\titlespacing{\subsubsection}{0pt}{0.5ex}{0.5ex}
% 表标题
\usepackage{caption}
\captionsetup{font={small,bf,stretch=1.25}}


% 重新定义公式编号格式为章节号-公式序号

% 重新定义公式编号格式为阿拉伯数字章节号-公式序号
\renewcommand{\theequation}{\arabic{section}-\arabic{equation}}

% 在每个章节开始时重置公式编号
\counterwithin{equation}{section}

% 目录格式定制(加入"第X章")




\renewcommand\thesection{\arabic{section}}
\renewcommand\thesubsection{\arabic{section}.\arabic{subsection}}
\renewcommand\thesubsubsection{\arabic{section}.\arabic{subsection}.\arabic{subsubsection}}







%参考文献为上标
\newcommand{\ucite}[1]{\textsuperscript{\cite{#1}}}


%设置分数始终大小一致
\everymath{\displaystyle}


%matlab 引用
\usepackage{listings, matlab-prettifier}
\lstset{
	style     = Matlab-editor,
	numbers   = left,
	frame     = single,
}


%目录深度
%\setcounter{tocdepth}{2} 


\linespread{1.333} % 真正的1.5倍行距	







\begin{document}
	
	% 封面
	
	\begin{titlepage}
		\begin{flushleft}
			\vspace{0.5cm}
			{\zihao{5}\bfseries 中图分类号:O441.5 \\ 
				论文编号:10006SY1519121}
		\end{flushleft}
		
		\vspace{3cm}
		
		\begin{center}
			\vspace*{1cm}
			{\zihao{0}\bfseries 北京航空航天大学\\硕士学位论文} \\[1.5em]
			{\zihao{1}\bfseries 基于SQUID的磁显微测量微缺陷研究} \\[4em]
			
			\renewcommand{\arraystretch}{1.3}
			\begin{tabular}{@{}ll@{}}
				\zihao{4}\bfseries 作者姓名 & \zihao{4}\bfseries 李园园 \\[1em]
				\zihao{4}\bfseries 学科专业 & \zihao{4}\bfseries 无线电物理 \\[1em]
				\zihao{4}\bfseries 指导教师 & \zihao{4}\bfseries 王三胜\quad 副教授 \\[1em]
				\zihao{4}\bfseries 培养院系 & \zihao{4}\bfseries 物理科学与核能工程学院 \\[1em]
			\end{tabular} \\[4em]
			
			%{\zihao{4}\bfseries \today}
		\end{center}
		
		\vspace*{\fill}
	\end{titlepage}
	
	\newpage%英文呢标题
	\begin{titlepage}
		
	
	\begin{center}
		\vspace*{3cm}
		{\fontsize{18pt}{22pt}\selectfont Research on Micro-defect Detecting\\ 
			by Magnetic Measuring based on SQUID}
		
		\vspace{4cm}
		
		{\fontsize{12pt}{14pt}\selectfont A Dissertation Submitted for the Degree of Master}
		
		\vspace{4cm}
		
		{\fontsize{15pt}{18pt}\selectfont Candidate:Li Yuanyuan \\
			
			Supervisor:Associate Prof. Wang Sansheng}
		
		\vspace{4cm}
		
		{\fontsize{15pt}{18pt}\selectfont \textbf{School of Physics and Nuclear Energy Engineering\\  
				Beihang University,Beijing,China}}
		
	\end{center}
	
	\end{titlepage}
	
	\newpage
	
	\begin{titlepage}
		\begin{flushleft}
			\vspace{0.5cm}
			{\zihao{5}\bfseries 中图分类号:O441.5 \\ 
				论文编号:10006SY1519121}
		\end{flushleft}
		\begin{center}
			\vspace{2.5cm}
			
			{\zihao{-2} \heiti 硕  士  学  位  论  文}
			
			\vspace{2.5cm}
			
			{\zihao{-1} \heiti 基于SQUID的磁显微测量微缺陷研究}
			
			\vspace{5cm}
			
		
			\hspace*{-1.5cm}
			\zihao{4} % 表格内所有文字为四号字
			\setlength{\tabcolsep}{10pt} % 单元格间距
			\begin{tabular}{ll@{\hspace{1em}}ll@{}} % 标准四列布局
				作者姓名 & 李园园 & 申请学位级别 & 理学硕士 \\[1em]
				指导教师姓名 & 王三胜 & 职 称 & 副教授 \\[1em]
				学科专业 & 物理学 & 研究方向 & 磁场测量技术研究 \\[1em]
				学习时间自 & 2015年9月1日 & 起至 & 2018年3月1日止 \\[1em]
				论文提交日期 & 2017年12月28日 & 学位授予单位 & 北京航空航天大学 \\[1em]
				论文答辩日期 & 2018年1月17日 & 学位授予日期 & 年 月 日 \\[1em]
			\end{tabular} \\[2em]
			
		\end{center}
		
		
		
		
	\end{titlepage}
	
	\newpage
	\thispagestyle{empty} % 去除页眉页脚
	
	% 关于学位论文的独创性声明部分
	\begin{center}
		\zihao{3}\bfseries 关于学位论文的独创性声明 \\[2em]
	\end{center}
	
	\zihao{5}
	本人郑重声明:所呈交的论文是本人在指导教师指导下独立进行研究工作所取得的成果,论文中有关资料和数据是实事求是的。尽我所知,除文中已经加以标注和致谢外,本论文不包含其他人已经发表或撰写的研究成果,也不包含本人或他人为获得北京航空航天大学或其它教育机构的学位或学历证书而使用过的材料。与我一同工作的同事对研究所做的任何贡献均已在论文中做出了明确的说明。 \\[1.5em]
	
	若有不实之处,本人愿意承担相关法律责任。 \\[2em]
	
	\begin{flushleft}
		\zihao{5}学位论文作者签名:\rule{5cm}{0.15mm} \hfill 日期:\hspace{1.5em}年\hspace{1.5em}月\hspace{1.5em}日
	\end{flushleft}
	
	\vspace{3em} % 适当增加间距
	
	% 学位论文使用授权书部分
	\begin{center}
		\zihao{3}\bfseries 学位论文使用授权书 \\[2em]
	\end{center}
	
	\zihao{5}
	本人完全同意北京航空航天大学有权使用本学位论文(包括但不限于其印刷版和电子版),使用方式包括但不限于:保留学位论文,按规定向国家有关部门(机构)送交学位论文,以学术交流为目的赠送和交换学位论文,允许学位论文被查阅、借阅和复印,将学位论文的全部或部分内容编入有关数据库进行检索,采用影印、缩印或其他复制手段保存学位论文。 \\[1.5em]
	
	保密学位论文在解密后的使用授权同上。 \\[2em]
	
	\begin{flushleft}
		\zihao{5}
		学位论文作者签名:\rule{5cm}{0.15mm} \hfill 日期:\hspace{1.5em}年\hspace{1.5em}月\hspace{1.5em}日 \\[1em]
		指导教师签名:\rule{5cm}{0.15mm} \hfill 日期:\hspace{1.5em}年\hspace{1.5em}月\hspace{1.5em}日
	\end{flushleft}
	
	
	\newpage
	\pagestyle{plain}
	\pagenumbering{roman}
	\section*{摘要}	
	\zihao{-4}
		\phantomsection\addcontentsline{toc}{section}{摘要}\tolerance=500 %将摘要放进目录
	无损检测技术发展水平的高低标志着一个国家基础工业发展的好坏,它作为现代工业发展中的基础性技术,能够有效地保证工业产品的品质和工程项目的质量,被广泛誉为工业界的“质量卫士”。\par
	集成电路制造商目前已采用多种方法和技术来定位布线缺陷,包括光学显微镜、扫描电子显微镜、扫描探针显微镜和微探针。也许在诊断IC短路问题中,最尖端技术是利用热成像与红外线摄影机相结合方法。当电路行为没有明显的缺陷时,这对于定位缺陷尤其有用。最明显情况是短路问题,因为短路会引起大量的电流,产生一个“热点”在热图像。但是,热成像的空间分辨率很低,特别是装有“倒装芯片”的集成电路,因为厚厚的导热基板。而超导量子干涉仪(SQUID)作为一种高灵敏度、高空间分辨率的磁传感器,可以在大的均匀场环境下检测fT量级的微小磁场信号,是对复杂电路板内部缺陷进行探测和确定的有效途径。\par
	本硕士论文旨在通过系列深入地研究,以电磁学磁场分布为理论指导,以微弱磁场分布检测为依托,进而引入数值分析方法,最终实现对微弱磁场信息的有效检测、精确计算与反演分析。通过对SQUID工作原理、电磁场分布理论和电磁反演理论的研究和分析,为后续的技术实现、系统搭建和实验验证提供理论指导;通过对高效磁屏蔽理论的分析和技术实现,提供近零磁场环境,使SQUID进入正常工作模式;通过磁聚焦器的研究和实现,将磁聚焦器与磁传感器相耦合,最大限度的提高了SQUID磁传感器的空间分辨率;结合上述理论研究与技术实现,进一步搭建了基于SQUID的磁显微测量微缺陷原理样机,该样机能够自动化地、高灵敏度地实现对被测对象磁场分布的检测;之后,结合检测数据的磁场信号特征,研究分析了行之有效的数据分析与处理方法,及检测结果云图绘制算法,确保检测结果的准确、直观和高效。最后,利用之前搭建的实验装置,对典型缺陷电路板进行检测,实验结果以平面云图形式直观表明缺陷位置与类型,完成本论文的研究目标。
		\\ \textbf{关键词:}SQUID、磁场测量、主被动磁屏蔽、空间分辨率、缺陷检测
		
		
	\newpage
		
	\section*{Abstract}
	
	
		\phantomsection\addcontentsline{toc}{section}{Abstract}\tolerance=500 %将 Abstract 放进目录 
		
		{\fontsize{12pt}{15pt}\selectfont
		
		As the foundation of the development of modern industrial technology, non-destructive testing technology plays an important role in ensuring the quality of industrial products and engineering projects. It is also widely known as the industry's "quality guardian", which can measure the advanced degree of the national basic industry.\par
		Traditional non-destructive testing techniques mainly include thermal imaging, ultrasonic, magnetic powder and single/multi-frequency eddy current testing, etc. These technologies can play a role in a particular defect-detecting field, effectively figuring out the presence of surface defects, also completing the corresponding quantitative evaluation. However, they can’t help detecting the tiny or deep cracks in multi-layer composite structures. Superconducting quantum interference device (SQUID) has the characteristics of high sensitivity and high spatial resolution which can detect the tiny magnetic signal of fT magnitude around a large homogeneous magnetic field. As a result, it can effectively detect the internal defects or corrosion crack in multilayer material, which is regarded as an effective way to solve the above problems.\par
		On the deeply research of electromagnetic field distribution theory, this paper aims to realize the effective detection, accurate calculation and successful inversion analysis of tiny defects in samples, by weak magnetic field information detecting, which achieves non-destructive testing. Firstly, sufficient research and analysis of the principle of SQUID, electromagnetic field distribution theory and electromagnetic inversion theory will be carried out to provide theoretical guidance for subsequent technical implementation, system building and experimental verification. Secondly, to ensure the SQUID's efficient working mode, it will research and analyze the magnetic shielding technique, and then set up a suitable magnetic shielding system for SQUID. In addition, realization of the magnetic focuser and 3d micro-displacement control system will improve the spatial resolution of the SQUID magnetic sensor. With the above results of theory research and technology achievement, it will further set up the prototype of micro-defect detecting system based on SQUID, which can automatically and highly sensitively realize the magnetic field distributing detection of the measured objects. Furthermore, combining with the characteristics of the testing data from the prototype, it will analyze the effective methods of data analysis and processing, and design the data-nephogram rendering algorithms, which make sure that the test results are accurate, intuitive and efficient. Finally, according to the above theory research and hardware system, some typical circuit boards will be as samples to carry out a series of experiments. The results can successfully detect the defects, showing out the abnormal curve plane and plane-scanning clouds intuitively. \par
		In a word, the study of this paper can preliminarily realize the defect-detecting mission, achieving the goals of the research task.
		\\ \textbf{Keywords: }SQUID, Magnetic-field Measuring, Active and Passive Magnetic Shielding, Spatial Resolution, Defect Detecting.
	}
	% 目录
	\newpage
	\tableofcontents




	
	
	
	
	\newpage
	
	%=======================================
	% 正文设置
	\pagenumbering{arabic}
	% 页眉页脚
	%\pagestyle{fancy}
	%\fancyhead[L]{}
	%\fancyhead[C]{\zihao{-5}中间页眉小五号}
	%\fancyhead[R]{}

	
	% 设置页面样式
	\pagestyle{fancy}
	\fancyhf{} % 清除所有页眉页脚
	\fancyfoot[C]{\thepage} % 页脚居中显示页码
	
	% 存储当前章节标题的命令
	\newcommand{\currentsectiontitle}{}
	
	% 定义一个计数器来判断奇偶页
	\newcounter{mypagecounter}
	\setcounter{mypagecounter}{1}
	
	% 重定义\section命令,捕获标题并更新标记
	\let\oldsection\section
	\RenewDocumentCommand{\section}{som}{%
		\IfBooleanTF{#1}
		{\oldsection*{#3}%
			\gdef\currentsectiontitle{#3}% 无编号节
		}
		{\IfValueTF{#2}
			{\oldsection[#2]{#3}%
				\gdef\currentsectiontitle{第\chinese{section}章\ #3}% 有编号节,带短标题
			}
			{\oldsection{#3}%
				\gdef\currentsectiontitle{第\chinese{section}章\ #3}% 有编号节,无短标题
			}%
		}%
		% 更新页眉
		\thispagestyle{fancy}%
	}
	
	% 设置页眉(单页模式)
	\fancyhead[C]
	{\ifodd\value{mypagecounter}{\zihao{-5}\textbf{北京航空航天大学硕士学位论文}} 
	 \else {\zihao{-5}\textbf{\currentsectiontitle }}\fi}
	 
	  % 根据计数器交替显示内容
	%\fancyhead[R]{\textbf{特定页眉内容}} % 右侧显示特定页眉内容(可以根据需要调整)
	
	% 在每页更新计数器
	\AddToHook{shipout/background}{%
		\stepcounter{mypagecounter}%
	}

	
	
	
	% 正文开始 arabic 编号
	
	
	
	%=======================================

	\section{绪论}
	
	\subsection{课题背景及意义}
	无损检测技术发展水平的高低标志着一个国家基础工业发展的好坏,它作为现代工业发展中的技术基础研究,在检测产品质量和医疗应用上占据着重要的地位。从航空航天工业到核能工程领域,针对诸如裂纹、腐化等问题,表面和深层缺陷检测是一项非常重要的研究课题[1-4]。另外,在民用和工业领域,可以用于大型集成电路生产流水线上的在线监测,,有助于降低生产成本[5-10]。除此之外,在生物医学领域中,脑磁检测是一种无创检测技术,它可以测量人类大脑头部由神经元活动产生的表面磁场,也可以同时记录人类大脑中核磁共振质子信号,这种对大脑无损害的临床医疗检测设备的研究更是国际上的前沿课题[11-18]。\par
	随着各类电子系统集成度的提高,芯片中的互联结构也变得更加复杂。集成电路制造商目前已采用多种方法和技术来定位布线缺陷,包括光学显微镜、扫描电子显微镜、扫描探针显微镜和微探针。也许在诊断IC短路问题中,最尖端技术是利用热成像与红外线摄影机相结合方法。当电路行为没有明显的缺陷时,这对于定位缺陷尤其有用。最明显情况是短路问题,因为短路会引起大量的电流,产生一个“热点”在热图像。但是,热成像的空间分辨率很低,特别是装有“倒装芯片”的集成电路,因为厚厚的导热基板。为了解决上述问题,需要借助具有高灵敏度与高空间分辨率的磁传感器,实现fT量级的磁场测量精度,从而实现对电路典型缺陷的有效检测。\par
	超导量子干涉仪(SQUID)是一种高灵敏度的磁传感器,其空间分辨率达到um量级,在深层缺陷和微缺陷检测领域具有其他技术无法比拟的优势,故而在目前的无损检测技术中占据至关重要的位置,而该技术在国内的发展目前才刚刚开始,因此相关研究学者应该大力加强对它的深入研究和推广应用,本毕业论文中所阐述的相关工作便是对该技术一定程度的探索、研究与应用成果。\par
	\subsection{国内外研究现状}
	
	\subsubsection{国际研究进展情况}
	A. Mathai等人在1993年使用4.2K Nb—PbIn直流SQUID构造了一种新型的一维磁通显微镜,样品与传感器都为低温环境下,它的空间分辨率和磁场分辨率是前所未有的。在成像过程中,样品与传感器之间距离大约在38。该系统实现了66左右的空间分辨率和磁场分辨率大约5.2pT在6千赫的频率。我们利用显微镜获得了低场超导样品的磁化率图像,并设计了一种测量静态磁场的简单方法[19]。\par
	美国马里兰大学用一个高温扫描SQUID显微镜在室温下成像半导体电路。我们的显微镜使用市售的77 K制冷机冷却直流SQUID。该系统保持真空隔离,它与从室温样品分离约30。当以这种方式操作,SQUID磁场灵敏度为有20 pT 在500赫兹下。通过磁场图像反演为二维电流密度分布,当前空间分辨率定位到36时,样品与传感器间隔为150时。该系统测量装置如图1(a)所示,在扫描过程中,SQUID保持在样品上方的固定高度,SQUID的平面平行于样品,因此SQUID测量磁场的z分量。对一个在两个引脚之间有一个死短的微电子芯片进行磁成像,测量结果如图(b)所示,这显然是短路的地方,因为所有的电流都必须通过设备中的单一缺陷,实现了对10cm×10cm大样品的有效检测,空间分辨率为150[8]。\par
	
	\begin{figure}[htbp]
		\centering
		\begin{subfigure}{0.45\textwidth}
			\includegraphics[width=\textwidth]{图1-1}
			% \caption{子图1的说明} % 如果不需要子图标题,可以注释掉这一行
			%\label{fig:subfig1}
		\end{subfigure}
		\hfill % 添加水平填充,使两个子图之间有间隔
		\begin{subfigure}{0.45\textwidth}
			\includegraphics[width=\textwidth]{图1-2}
			% \caption{子图2的说明} % 如果不需要子图标题,可以注释掉这一行
			%\label{fig:subfig2}
		\end{subfigure}
		\caption{美国马里兰大学HTS SQUID显微镜实物图及检测结果}\label{图1}
	\end{figure}
	
	
	L.A.Knauss等人实现了室温下封装水平芯片缺陷检测,系统灵敏度达到20pT,空间分辨率达到50,图2为芯片级别电源短路检测图,图中亮点部分为短路缺陷发生的位置[10]。\par
	
		\begin{figure}[htbp]
		\centering
		\includegraphics[width=0.7\textwidth]{图2}
		\caption{磁场分布云图识别芯片短路缺陷}\label{图2}
		\end{figure}
	
	2004年,日本搭建了一个用于检测食品中金属污染物的高温超导量子干涉仪系统[5, 6],它能成功地检测到0.6到0.1毫米大小的不锈钢球和1毫米大小的铜球,SQUID检测结果与理论计算值相吻合(如图3),检测水平优于当时的市场需求。\par
	随着社会的不断发展,为了提高该技术在工业现场的可适用性,良好的磁屏蔽技术成为其进一步发展的关键。2009年,日本开发了两种基于双通道高温超导量子干涉仪,面向工业产品和饮料的污染物检测系统,其检测目标异物的颗粒大小为小于100 um和300um。该系统借助带式输送机,可成功地检测到水中100um (4.12ug)的小铁颗粒和300um(120ug)不锈钢球。\par
	借助超导量子干涉仪搭建的上述诸多检测设备,在缺陷检测领域所达到的水平是常规X射线检测或其他方法很难实现。\par
	\begin{figure}[htbp]
		\centering
		\begin{subfigure}[b]{0.45\textwidth}
			\includegraphics[width=\textwidth]{图3-1}
			%\caption{子图A}
		\end{subfigure}
		\hfill
		\begin{subfigure}[b]{0.45\textwidth}
			\includegraphics[width=\textwidth]{图3-2}
			%\caption{子图B}
		\end{subfigure}
		\caption{用于食品污染物检测的高温超导量子干涉仪系统与检测结果}\label{图3}
	\end{figure}
	
	\subsubsection{国内研究进展情况}
	我国已成功研制了室温扫描 SQUID显微镜,被称为高温 SSM 系统,其结构如图4所示。其中,图(a)为扫描SQUID显微镜杜瓦结构,主要由无磁不锈钢、黄铜和蓝宝石片组成;图(b)为杜瓦底部详细图。该显微镜采用高温直流SQUID作为传感器,将没有接收线圈的SQUID超导环单独贴在黄铜冷指上,进而透过厚度为75μm的蓝宝石窗口对室温样品进行检测。另外,该系统通过调节波纹管使SQUID环直接贴近蓝宝石窗口,而没有使用接收线圈的磁通耦合形式,实现了150的空间分辨率和46 pT的磁场灵敏度,能够实现对被测样品的缺陷检测[20-22]。\par
	\begin{figure}[htbp]
		\centering
		\includegraphics[width=0.7\textwidth]{图4}
		\caption{中科院物理所室温样品 SSM 结构示意图}\label{图4}
	\end{figure}
	目前,国内其他关于扫描SQUID显微镜硬件系统的研究都是围绕上述系统的某个部分展开。2005年,重庆大学的钟超荣等人对该系统空间分辨进行了改进研究,引入了磁聚焦技术[23],但仅限于仿真阶段。2008年,北京科技大学的江忠胜利用该系统进行了初步无损检测实验研究,效果差强人意。\par
	另外,北大超导实验室利用高温超导YBCO薄膜材料研制成了射频超导量子干涉仪磁强计系统,系统灵敏度达到了优于80fT/的国际先进水平。该实验室借助该系统首次开展了对动物心磁的实验研究,取得了一些突破性进展。\par
	综上所述,目前国际上基于SQUID的无损检测方面的工作技术相对比较成熟,而国内在这方面的探索和研究与国外有许多的差距,需要加大力量进行推进和不断的完善。\par
	\subsubsection{技术发展趋势}
	由于SQUID器件具有高灵敏度的特征,其磁场检测技术区别于传统射线检测,是利用高灵敏度SQUID磁传感器对样品进行扫描成像,从而实现在线完整性诊断的新兴技术,该技术可检测到被测构件结构更多的特征信息,从而更加准确、详细的反应其结构特征,完成缺陷检测的任务。\par
	基于SQUID的微缺陷检测技术在实际应用中需要尽可能的发挥SQUID在磁检测方面的优势,尽可能的减小外界环境对其性能的影响,充分发挥其高灵敏度的特点。另外,针对微缺陷的应用方向,如何直观、高效地显示检测结果也是该技术在工程应用中所面对的关键问题。与此同时,针对微缺陷检测的基于SQUID的完整磁检测系统的搭建和有效运行是该技术应用的基础。\par
	综上所述,该项技术具有以下四方面的发展趋势:
	\begin{enumerate}
		\item[a)] SQUID高空间分辨率技术实现研究;
		\item[b)] 高效磁屏蔽技术研究;
		\item[c)] 快速便捷的扫描磁检测系统与磁成像技术研究;
		\item[d)] 基于SQUID的微缺陷检测系统的样机搭建与应用实现。
	\end{enumerate}

	\subsection{论文主要研究内容及安排}
	本论文旨在首先通过系列深入地研究,以电磁学磁场分布理论为指导,借助数值分析方法,实现微弱磁场无损检测问题中磁场信息的精确计算与分析。另外,基于SQUID应用的高效磁屏蔽技术和磁聚焦技术的研究,有助于搭建空间分辨率为um量级的磁场测量原理样机。在此基础上,需要进一步研究缺陷检测的磁场信号特征,选择合适的特征信号,并引入合理的滤波处理与反演方法,从而获取被测对象中缺陷存在的详细信息,最终实现面向微弱缺陷的无损检测。\par
	本文共分为五章,内容介绍如下:\par
	第一章,绪论。主要对基于SQUID的磁显微测量微缺陷的研究必要性进行简要阐述。之后,对国内外SQUID微缺陷检测研究情况进行概述,并总结了下一步该技术的发展趋势。\par
	第二章,SQUID介绍及磁场检测理论研究。基于本论文所使用的SQUID磁传感器,首先研究介绍了SQUID的设计原理、工作条件及传感器系统组成。另外,结合磁场检测的目标,研究了磁场分布的基础理论——毕奥-萨伐尔定律,进一步分析了磁场分布的电磁反演理论,为后续的实验应用和技术实现提供理论指导。\par
	第三章,磁屏蔽技术研究。鉴于SQUID磁传感器对于磁场信号的高灵敏度,磁屏蔽技术对其实验应用的成败起到关键作用。本章主要研究了传统被动屏蔽和超导屏蔽技术的原理,并总结了其局限性;进一步地提出了一种主动屏蔽与超导屏蔽相结合的方法,既实现了实时补偿,又有效地抑制了交变磁场和多方向磁场的干扰,大大提高了磁场的屏蔽效果,保证了高灵敏度SQUID在微弱磁场测量中的成功应用。 \par
	第四章,基于SQUID磁聚焦器的研究与设计。空间分辨率的高低是SQUID应用过程中的一大难题,本章意在通过磁聚焦器的研究与设计实现对SQUID空间分辨率的优化。通过一系列的理论研究,重点分析了多个参数对磁化现象的影响,并借助有限元仿真分析手段进行了验证,最终设计出最优参数化的磁聚焦器,大大提高了空间分辨率。\par
	第五章,基于SQUID的磁显微测量微缺陷系统搭建与实验分析。本章主要基于上述理论分析与技术实现的结果,搭建了基于SQUID的磁显微测量微缺陷原理样机,包括:SQUID磁传感器部分、磁屏蔽部分、三维微位移扫描平台部分和上位机数据分析与处理部分。并基于该样机的实验数据,进一步地研究分析了对检测数据的滤波算法和结果云图绘制算法,以期直观、有效的获取磁检测结果。最后,以典型电路结构为检测对象,进行了实验缺陷检测。结果表明,该原理样机能够实现微缺陷检测的目的,达到了本论文的研究目标。\par
	最后,总结了本硕士学位论文在基于SQUID的磁显微测量微缺陷系统研究领域所完成主要工作,并该项技术下一步的应用前景进行展望。\par
	
	
	\newpage 
	\section{SQUID及磁场检测理论研究}
	
	\subsection{引言}
	由于SQUID磁传感器对于磁场信号的高灵敏度,可以达到fT量级,因此,微弱磁场测量实验中必须使用该传感器,本论文的研究课题便是充分利用SQUID器件,实现对被测构件中微缺陷弱磁信号的有效检测。\par
	基于上述研究目标,本章首先介绍了SQUID的工作原理,为后续的实验应用做准备;其次,作为本论文的根本检测对象,详细研究了磁场分布与电磁反演理论,为后续实验检测过程中的数据处理和结果展示提供理论指导。\par
	\subsection{超导量子干涉仪工作原理}
	超导量子干涉仪(SQUID)是超导电子学技术中的基本元件,这种器件在恰当的工作条件下可以将外部磁场变化高效地转换为电压信号。目前,优化设计的HTC SQUID传感器能够分辨的磁通变化为,而由于磁通量子w,本身就是一个很小的物理量,因此本质上可使用 SQUID磁强计实现对高精度磁场的测量。\par
	SQUID传感器最基本的特征是输出电压对外部磁场的变化表现出一种以磁通量子为周期的函数关系[24-26]。为了增加测量灵敏度和线性化输入/输出响应,SQUID磁强计的设计通常采用属通锁定工作模式,在这种工作模式中,变化的待测磁场使进入 SQUID环孔中的磁通发生相应变化(表示SQUID传感器的有效面积,是其重要的设计指标),该磁通变化被 SQUID传感器转换成电压变化交给磁通锁定电路进行放大、解调、积分和电压/电流变换,最终通过一个与SQUID耦合的电感在SQUID环孔中产生一个反向磁通以抵消外环孔中的磁通变化,其输出信号就是与△Φ成正比的积分器输出电压。\par
	上述设计模式(工作模式)除了能够提供正比于待测磁场变化的输出电压信号,还能把前置放大器的低频噪声、电路系统的温漂和SQUID的器件参数漂移对磁强计系统的影响降到最小。 \par               
	本论文采用XY-2型高温超导 SQUID磁强计,主要由 SQUID传感器、磁通锁定电路和SQUID控制器3部件组成,下图为其工作原理框图。\par
	
	\begin{figure}[htbp]
		\centering
		\includegraphics[width=0.7\textwidth]{图5}
		\caption{高温超导 SQUID磁强计工作原理框图}\label{图5}
	\end{figure}
	
	XY-2型高温超导SQUID磁强计所使用的SQUID传感器采用临界转变温度高达90K的高温超导材料(YBa2Cu3O7X)制备,其核心技术是台阶边沿约瑟夫森结技术、大面积磁聚焦器/高谐振频率介质谐振器技术和SQUID芯片与磁聚焦器/超导介质谐振器组件的组合封装技术。其中,磁通锁定电路采用无磁通调制式锁定电路设计;SQUID控制器则由多种功能的外围电路组成,通过对谐振器频率、射频偏置功率和直流电平等参量的控制以保证SQUID传感器能在不同环境干扰、背景磁场下处于最佳偏置和最佳磁通锁定工作状态。\par
	\subsection{磁场分布理论}
	本论文进行磁场检测与反演所依据的基本原理是毕奥-萨伐尔定律,其公式如(2.1)所示,原理示意图如图 6 所示[27]。\par
	\begin{equation}
		dB = \dfrac{\mu_{0}}{4 \pi} \dfrac{IdI\times e_{r}}{r^{2}}
	\end{equation}
	
	\begin{figure}[htbp]
		\centering
		\includegraphics[width=0.35\textwidth]{图6}
		\caption{点电流产生磁场示意图}\label{图6}
	\end{figure}
	其中,B是磁感应强度,单位:T;是真空磁导率;I代表电流,单位:A;\par
	上式大小:$dB = \dfrac{\mu_{0}}{4 \pi} \dfrac{IdI\times e_{r}}{r^{2}}$;\par
	方向:$IdI\times e_{r}$;\par
	通过上述过程,可实现对被测构件中微缺陷磁场分布情况的理论分析与计算。

	
	\subsection{磁场分布的电磁反演}
	对于基于SQUID的磁显微测量微缺陷研究,其关注的主要磁场信息量为被测构件中缺陷磁信号在空间Z方向的分量,该磁场信息的电磁反演事关检测结果的正确性,是本课题的重要研究内容。
	
	
	
	\subsubsection{傅里叶变换和傅里叶逆变换}
	根据毕奥-萨伐尔定律,z方向磁场可由x,y方向电流密度求出,即[28, 29]
	
	\begin{equation}
		B_{z} = \dfrac{\mu_{0}}{4 \pi} \int_{0}^{a}\int_{0}^{b}
		\dfrac{(y-y^{'}) J_{x}-(x-x^{'}) J_{y}}{[(x-x^{'})^{2}+(y-y^{'})^{2}+z^{2}]^{\frac{3}{2}}}
	\end{equation}
	对公式(2.2)进行傅里叶变换:\par
	
	
	其中,k为傅里叶空间变量。\par 
	再由电流空间的连续性得到电流密度的散度为0,即\par
	上述公式(2.2)和(2.3)中x,y方向的电流密度都是线性的,联立两式可得到频域空间的电流密度,如下:\par
	
	
	最后,对频域空间的电流密度进行傅里叶逆变换,便可得到时域空间的电流密度。\par
	对采集的磁场采用傅里叶变换、逆变换的方法来反演求解电流,可得到电流分布图。该方法特点为模型简单、数据处理与编程较容易,但经过傅里叶变换与逆变换处理过的电流分布中,电流信息缺失严重,并不能与磁场分布中的“缺陷”相对应。产生不匹配的原因可能是因为这种方法经过空间域和空间频上的变化后,局部的磁场变化会影响到整个电流空间分布,而且影响到整个二维平面的电流分布。该方法并未达到最理想效果,为了消除这些影响,需要进一步尝试引入滤波算法。
	
	\subsubsection{吉洪诺夫正则化}
	无限长薄电流片产生的磁场在z方向上的分量为[30-33]:\par
	
	
	数学计算上,沿带材电流垂直方向将带材表面分割成若干电流微元,然后根据毕奥-萨伐尔定律沿电流垂直方向的测量数据,列出线性方程组,再对线性方程组进行反演,便可求出电流分布。该方法要求测量的磁场数值与真实值差别不能太大,不然会造成求解得到的各电流微元上的电流极大地偏离真实值。\par
	病态方程组的求解主要采用两种方法:第一种方法是使用双精度的高斯主元消去法或者LU分解的数值法。单精度的数值计算最大能得到5阶矩阵的线性方程组,而当矩阵高于5阶,单精度的算法就得到正确的结果。如果需要得到20-30阶的线性方程组的稳定数值解,应该利用双精度计算。\par
	此外,更高精度的方法为正则化方法。以Tikhonov为代表的数学家于二十世纪六十年代提出了正则化方法,其基本思想是:通过引入正则化,使病态问题变得稳定,从而构成一个与原问题相临近的适应性问题,并用它的解来近似估计原问题的解。\par
		
			图7\\
	如图7所示,我们沿电流传播方向将被测对象划分为若干等宽的矩形“细条”,在每个细条内近似认为电流密度相等,若能通过磁场反演得知各矩形细条上的电流大小,就可以得知被测对象的表面电流分布。\par
	使用Tikhonov正则化方法处理,需要构建比较复杂的物理模型:沿电流传播方向将被测对象划分为若干等宽的矩形,i表示矩形的空间位置、k表示霍尔探头的空间位置,则位于k处的霍尔探头测得的磁感应强度是被测对象上每个矩形上的电流在k处产生的磁场的叠加。根据(2.7)式,建立如下方程:\par
	
	其中,Bz(k,i)表示i处的矩形在k处产生的磁场z轴方向的分量,z为扫描霍尔探测器探头到被测对象所在平面的距离,yk为霍尔探头在y方向上的位置,yi为矩形在y方向上的位置,b为矩形宽度,真空中的磁导率,J(i)为i处矩形上的电流密度。\par
	
	考虑到位于K处的霍尔探头测得的磁感应强度是超导带材上每个矩形上的电流分别在K处产生的磁场的磁感应强度的叠加,可以写出线性方程组:\par

	通过求解线性方程组的逆矩阵,就可以得知被测对象的表面电流分布。因为该问题常为不适定的,可以通过Tikhonov正则化方法来处理,(2.11)式的求解就转化为适定题目的求解。\par

	其中,正则化参数是,用来控制与的相对大小;为Tikhonov正则化矩阵,其具体形式与具体问题中矩阵的具体形式有关,一般取为单位阵。通过添加等号右边第二项,将不适定问题正则化,再用多次迭代的方法来求解。\par
	通过在等号右边残差范数后面加上一项,(2.11)式中的不适定问题便被正则化了,(2.12)式的解是(2.11)式的解的一个很好的近似,而且满足解的适定性条件,可以很方便地通过对(2.12)式的迭代来求解。此即吉洪诺夫正则化方法的基本思路。

	
	\subsection{本章小结}

	本章主要讨论 SQUID 的工作原理和系统组成,并针对被测对象,研究了磁场分布理论。之后,深入研究分析了电磁反演方法。\par
	其中,SQUID工作原理方面,详细介绍了SQUID正常工作所需要的系统组成,从SQUID传感器、磁通锁定电路和SQUID控制器三个方面分布阐述,为后续系统的搭建和实现提供理论支撑和技术指导;\par
	磁场分布理论方面,则基于毕奥-萨伐尔理论,得出了基于SQUID的磁显微测量微缺陷磁场的磁场分布情况,从理论计算的角度得出被测对象的磁场分布与大小,可供实验检测验证与对比分析;\par
	电磁反演方法基于磁场分布理论,主要讨论了从被测结果到磁场分布的逆向计算过程,是本研究课题得已应用实现的最后环节,本章详细分析了傅里叶变换/逆变换和吉洪诺夫正则化两种方法,为最终是实验应用奠定理论基础。
	
	
	\newpage
	\section{磁屏蔽技术研究}

	\subsection{引言}
	在过去的几年里,已经有越来越多的行业领域利用SQUID进行微弱磁信号测量,例如:心磁、脑磁和对金属材料及样品的缺陷检测。心磁图是通过测量心脏产生的磁场来记录磁场,已经得到了广泛的研究。在微弱磁场无损检测中,被测样品磁场信号为nT量级,而正常环境下磁场信号为50000nT,因此,在超低磁场环境测量中,屏蔽环境电磁噪声是非常重要的。\par
	高性能的磁屏蔽有以下几种方法:(1)传统高磁导率材料的被动磁屏蔽方法;(2)基于高温超导的磁场屏蔽技术;(3)主动动态补偿磁屏蔽方案;(4)由本课题组提出的动态补偿磁屏蔽和基于高温超导线圈相结合的主被动磁屏蔽装置。\par
	本章将重点介绍主被动相结合磁屏蔽装置,并对其进行屏蔽因数测试。
	\subsection{高磁导率材料的被动磁屏蔽}
	传统的磁屏蔽通常是基于高磁导率的软磁材料组成,对于静磁屏蔽技术并不是“躲避”开磁场,而是为了吸引磁场,原理如图8所示。图中的壳体是用高磁导率材料(软磁材料)加工的,如坡莫合金、硅钢板和纯铁等,壳体就像一块吸铁石一样,把外部的磁场H0吸引在壳体上,该磁场无法摆脱壳体进入内部,所以壳体内剩余的磁场为Hi。(a)图给出了一种理想情况,即屏蔽壳体内部没有磁场进入;图(b)是现实的静磁屏蔽效果,有少量磁场进入了壳体。\par
	
	图8 \par
	
	评价屏蔽效果的参数为屏蔽因子 SF:\par
	
	因此,对于(a)的理想情况,屏蔽因子为无限大。
	
	\subsubsection{静磁场中屏蔽效果理论计算}
	
	\begin{enumerate}
		\item [(1)]单层结构\\
		对于球壳和无限长的圆柱壳轴向和横向的屏蔽因数进行分析,如果磁导率远远大于1,公式为[34]:\par
		其中,为外层直径,为内层直径,壳的厚度d=(),平均直径(),为材料的磁导率,一般会远远大于半径。\par
		对于轴向磁场方向的无限长圆筒,静态场没有屏蔽效率,纵向屏蔽系数SL= 1。对于有限长度,这种屏蔽系数可以用相同长度与直径比和相同磁导率的旋转椭球体来估计纵向屏蔽系数SL:\par
		
		其中,N为消磁系数。\par
		如果考虑到闭合圆柱的屏蔽效果,纵向屏蔽系数为\par
		
		
		
		\item [(2)]多层结构
		对于圆柱形磁屏蔽桶,轴向屏蔽因子与径向屏蔽因子是不相等的,而且轴向屏蔽因子往往会低于横向的屏蔽因子。对于多层圆柱形屏蔽桶的横向屏蔽因子,可以得到公式[34, 35]:\par
	\end{enumerate}
	
	\subsubsection{屏蔽桶设计}
	
	为了确定屏蔽筒相关设计参数对屏蔽性能的影响,首先采用有限元分析的方法对磁屏蔽性能进行分析研究。\par
	图9(a)给出了磁导率和磁屏蔽因数的仿真结果,观察可知磁屏蔽因数随着的增加,呈线性趋势增加,即磁导率越高,屏蔽因数越好。因此,在设计多层屏蔽桶时,为防止吸收磁饱和,最外层的磁导率要大于内层的磁导率。\par
	另外,针对屏蔽桶的层数对磁屏蔽效果的影响进行仿真分析,结果如图(b)所示。从图中可知,屏蔽桶每层之间是等间距的,层数越多,中心区域测得磁场值越小,屏蔽因数越高,故可通过增加层数提高屏蔽效果。\par
	图10(a)显示了屏蔽系数与层间隙之间的关系[36],从图中可知屏蔽系数随着层间隙的变化呈现出一种非线性关系,存在一个最优值。\par
	图9\par
	图10\par 
	
	进一步地,通过相关文献可知:Saburo Tanaka将套筒引入开放的屏蔽桶中[37, 38],以期获得更高的屏蔽系数。其研究考虑了四种套筒连接的位置情况:没有套筒、在开口的外桶上有套筒、在开口的内桶上有套筒和在开口的内桶和外桶上都有套筒。分析结果如图10(b)所示,结果可知SF随着套筒长度的增加而增大,在套筒40–60mm长处SF达到最大。最高屏蔽因子(SF = 2808)在最后一种情况中获得,比第一种情况下的屏蔽因数大2.3倍。上述研究结果表明,套筒在开放的屏蔽桶中起一定屏蔽作用。\par
	传统的磁屏蔽通常是由几层高磁导率材料和高电导率材料组合而成,这样的屏蔽系统是基于铁磁屏蔽和涡流屏蔽原理,可以很有效地减少环境磁场噪声,但是由于高成本和复杂的设计,往往不能满足实验要求,特别是野外环境。
	\subsection{基于高温超导材料的被动磁屏蔽}
	
	\subsubsection{理论分析}
	
	超导磁屏蔽的屏蔽原理与高磁导率屏蔽的屏蔽原理完全不同,超导磁屏蔽利用的是超导体的迈斯纳效应,由于超导体“不允许”其内部有任何磁场,如果外界有一个磁场要通过超导体内部,那么超导体必然会产生一个与之相反的磁场,保证内部磁场强度为零[39-41],这就会产生感应电流,该过程是自适应的,不消耗功率,不需外加电源。\par
	为了更好的分析超导的屏蔽能力,可建立简化为一个电阻串连电感,外加一个理想电压源的等效电路,在变化场下描述该电路的方程是[42, 43]:\par
	
	其中,N为线圈的匝数;L0是一匝线圈的电感;R是电路的电阻;A是线圈所包围的面积。\par
	(3.7)式的右边是由变化场引起的电势,可假设一个均匀的正弦波场,角频率ω,磁场峰值B0。当超导环处于超导状态,即R = 0时,由(3.7)式可得到每匝线圈的电流为:\par
	
	对式(3.8)的两个边乘以N,得到线圈的总的安培数为:
	
	\subsubsection{有限元仿真分析}
	引入有限元方法对超导环的屏蔽效果进行仿真分析,首先建立超导圆环的三维空间模型如图11(a)所示,外面的球为无限大的空气域,中心区域圆环为超导环。\par
	在设定边界条件和沿z轴激励源之后,获取最终的磁屏蔽计算结果,图(b)为横向磁通密度模分布图,从图中可得超导环内磁通密度模较小,边缘较大,分析原因为:超导环具有迈斯纳效应,产生感应电流来抵消内部磁场,增强外部磁场,所以内部磁场较小而边缘较大;图(c)为轴向磁通密度模分布图,从图中可知:在中心轴线上磁场密度模有最小值,磁力线被排斥在超导环外侧;图(d)分别为沿着直径方向与轴向方向的一维点线图,蓝色线为横向磁通密度模,从图中可知:沿着直径方向超导环内部磁通密度模比环外磁场小,在内部从边缘到圆心方向磁场逐渐增大,分析其原因为:远离超导环,即感应电流在内部产生的抵消磁场逐渐减弱,和外界磁场叠加以后,磁场值会变大。图中绿色线表示磁通密度模沿中心轴线的分布情况,即在超导环内磁通密度逐渐减小,中心位置为最小值。上述仿真结果可知,超导环对外界变化的磁场具有屏蔽作用。\par
	图11\par
	
	
	\subsection{主动动态补偿磁屏蔽}
	理论研究上,有两种类型的动态补偿系统,以负反馈为工作机制:电感式传感器系统和饱和磁通门系统。这两种类型的磁强计的传递特性是余弦,可进行矢量场的测量。\par
	基于主动动态补偿磁屏蔽的目的,本课题搭建了如图12所示的主动补偿系统,主要由三轴磁强计、高精度恒流源、三对方形亥姆霍兹线圈和控制计算机组成。其工作原理为:将具有高灵敏度的磁通门计放置在霍姆赫兹线圈中心,由其检测实时环境磁场值,并通过一定的算法计算出需要补偿该磁场所对应的电流值。进一步地,由控制计算机驱动恒流源向霍姆赫兹线圈施加同规格电流,以产生与环境磁场对应的补偿磁场,从而实现主动磁屏蔽的效果[44-46]。\par
	根据上述原理搭建的检测系统其工作步骤为:\par
	(1)设置零磁场,即当三轴磁通门磁强计用于检测XYZ分量上的磁场时,利用主计算机算法计算每一个需要抵消磁场的轴的电流值;
	(2)恒流源输出相应的电流到三轴线圈;
	(3)补偿后检测环境磁场值为零。该过程通过闭合负反馈回路实现,可补偿电流受周围磁场波动的影响。
	图13描述了反馈回路关闭和打开情况下,磁场的HX、HY与HZ分布。可以清楚地观察到闭环时,磁场值在0nT附近波动,而开环时磁场将发生漂移。因此,为了提高磁场的精度,减少周围磁场的干扰,应采用封闭的负反馈回路来控制该屏蔽系统。这种主动补偿可以人为达到10nT,最大屏蔽因数为58dB。\par
	图12\par
	图13\par
	
	
	\subsection{新型主被动磁屏蔽}
	三轴赫姆赫兹线圈可以屏蔽掉一些恒定和低频变化磁场,实现实时补偿,但是由于电源和复杂的反馈系统对中高频交变磁场实时跟踪屏蔽响应较弱,导致存在剩余磁场波动,对外磁场的屏蔽不完全,是主动屏蔽技术的一个极大的挑战。\par
	结合前文介绍的超导屏蔽原理可知,结合主动屏蔽动态补偿的优势和超导磁屏蔽技术可有效地克服上述问题。因此,本研究课题提出一种主被动屏蔽系统,由有源屏蔽系统和高温超导环组成,高温超导环放置在有源屏蔽区域的中心,并通过液氮冷却,结构如图14所示。\par
	在77 K温度下,系统被放置在均匀的磁场区域,采用高温超导量子干涉仪和高精度磁通门计对恒定外磁场和交变外磁场环境下的屏蔽特性进行了分析。\par
	引入屏蔽因数作为磁场屏蔽能力的有效量度,屏蔽因数SF为\par 
	其中,是无屏蔽时外部磁场值,在屏蔽区域内的磁场值,所有测量值在这里指的是z方向磁场值。\par 
	
	图14\par 
	 
	 \subsubsection{恒定外磁场下,屏蔽效能的分析}
	 图15为主被动磁屏蔽系统屏蔽磁场强度图,用三维磁通门计分别测量X、Y和Z三个方向上磁场值。可以看出,主被动磁屏蔽可以有效减少外界磁场,与X、Y轴磁场相比,Z轴磁场变化不大,Z轴磁场稳定在20nT。\par 
	 图15\par
	
	在超导环的中心区域,即Z=0(Z1=0)处,沿着横向x轴,以10毫米为间隔,均匀测量9个点。在有无超导环磁屏蔽的两种情况下(Z=0曲线表示没有超导环屏蔽条件下,Z1=0曲线表示有超导环屏蔽条件下),通过高精度磁通门传感器测得的SF结果如图16(a)所示。当超导环存在时,轴向屏蔽因子比无超导环时屏蔽因数大,最多达10dB。在中心区域,SF为最大值,沿着半径方向,SF逐渐减小。因此可知,超导环在z轴的屏蔽磁场非常有效。\par
	进一步对超导环的有效屏蔽区域进行研究,在不同高度处进行磁场测量,如图(b)所示。研究结果可知,在 Z1=0处,中心区域屏蔽因数最好且提供均匀的区域,沿着x轴,屏蔽因数逐渐下降。随着高度增加或降低,屏蔽因数均小于Z1=0处的屏蔽效果。因此,要把SQUID传感器固定在超导屏蔽中心。\par
	
	图16\par 
	进一步的,将超高灵敏度的SQUID放在屏蔽区域,只用三轴线圈屏蔽,不加超导环线圈的测试结果如图17(a)所示;只用超导线圈屏蔽,不加三轴线圈的测试结果如图(b)所示;三轴线圈和超导环一起屏蔽的测试结果如图(c)所示。测试结果可以得出:(c)图可以清楚看到施加的三角波,没有周围噪声的干扰,当按下复位按钮时,数字电压表上显示数值在0附近,此时SQUID进入磁通锁定工作状态,可以展开微弱磁场的测量;而(a)和(b)图中有明显的噪声干扰,磁场噪声超过SQUID的测量量程,SQUID不能进行正常工作。\par
	因此,可以证明该装置能够有效地屏蔽外界磁场干扰,使SQUID进入正常工作模式,不用搭建昂贵的磁屏蔽室和多层复合结构的磁屏蔽桶。该装置最明显的优势是:提供一个开放的磁屏蔽环境,方便样品扫描成像系统平台地搭建。\par 
	
	图17\par 
	\subsubsection{交变外磁场下,屏蔽效能的分析}
	 为了分析上述磁屏蔽系统在交变磁场环境下的屏蔽效果,本研究主动施加不同频率(10Hz、50Hz、100Hz、500Hz和1000Hz)与不同幅值(电压为1V、2V、5V和10V)的正弦波信号,经功率放大器放大后,输入到线圈中。在有无屏蔽时,借助交流高精度磁通门计检测交变磁场,其测试结果如下图18所示。\par
	 图中可知,相同电压下,低频信号产生磁场较大;随着电压增大,高低频信号产生磁场都相应变大,但是由于阻抗增大导致产生磁场变化较小。由图(a)和图(b)可以观察到,有屏蔽时测量磁场明显减小,10Hz 电压为10V 正弦波产生交变磁场,磁场测量值由无屏蔽时27V减小到13V,因此可以得出此屏蔽系统对高低频磁场信号都有屏蔽作用。\par 
	 图18\par  	
	 
	 进一步地,对不同幅值和频率交变磁场进行屏蔽效能分析,以SF表示屏蔽因数,SF=。由图19可知,总体来说,对低于500Hz信号,屏蔽效果较好,均在45dB以上,且屏蔽因数受幅值变化较小,此时与通常屏蔽装置相比,屏蔽效能较高。但对于高于500Hz信号,屏蔽效果在28dB以上,受幅值影响较大。综上考虑,在外界无任何屏蔽,且白天嘈杂城市环境中,整个屏蔽系统内部磁场均在100nT之内。该屏蔽系统,低频时几乎不受幅值影响,对低频屏蔽效果优于高频。\par
	 以频率为50Hz、电压为5V的正弦信号产生磁场为研究对象,分析磁场在z=0处沿x轴变化的情况,如图20所示。图中可知,在正负15mm 之间,为均匀区域,磁场值最小,屏蔽因数最大至67.7dB。在屏蔽区域边缘处,屏蔽效果有所下降,为66.5dB。可以看到,该屏蔽系统可以产生一个大面积的均匀屏蔽区域。\par
	 
	 图19图20\par 
	 
	 
	 \subsection{场冷与零场冷}
	 高温超导可以在零场冷和场冷两种情况下冷却,邓等人[47] 对超导块材的最大悬浮力在两种冷却方式下进行比较。实验结果表明,块材在零场冷时与场冷时悬浮力没有直接联系。\par
	 那两种冷却方式对超导体的屏蔽有什么样的影响,针对这一问题,到目前还没有直接相关的报道,因此有必要对两种冷却方式下高温超导体屏蔽效能进行比较。
	 \subsubsection{零场冷情况下,屏蔽效能的分析}
	 零场冷为高温超导体冷却进入超导态在无外磁场的情况下,即在屏蔽室近零磁场环境中冷却超导环,然后再分析其屏蔽效果。实验中,通过施加不同频率与幅值的正弦波信号,经功率放大器放大后,输入到线圈中,在有无屏蔽时,用示波器观察交变磁场的输出情况。\par
	 图21(a)给出了无外界磁屏蔽时的磁场测量结果图,磁场大小与输出电压成正比,因此我们用电压值代表相应的输出磁场幅值。观察到,随着电压增大,不同频率信号的输出电压都以相同趋势逐渐增大。在10V外界电压,10Hz情况下,输出电压最大为28V。在零场冷磁屏蔽装置中测试结果如图(b)所示,明显观察到,输出电压值减少,最大输出电压为6V。主被动磁屏蔽装置可以有效进行磁场屏蔽,屏蔽因数如图(c)所示,结论为低频信号屏蔽效果优于高频信号。\par 
	 图21\par 
	 
	 \subsubsection{场冷情况下,屏蔽效能的分析}
	 场冷为在外磁场存在的条件下冷却,直至达到超导态。实验设置,使超导环处于地磁场环境中,充分冷却后,施加不同频率与幅值的外界正弦信号,分析其屏蔽效果,测试结果如图22所示。\par
	 其中,图(a)为无屏蔽时磁场电压输出,可以看出:与场冷时结果大致一样;图(b)为场冷屏蔽时,磁场电压输出明显大于零场冷屏蔽结果。测量屏蔽因数结果如图(c)所示,场冷情况下,屏蔽因数减小,因此屏蔽能力下降。零场冷屏蔽因数在70dB左右降低到60dB以下。故可推断出,场冷与零场冷对屏蔽效果具有很大的影响,我们要尽量采取零场冷的方法进行超导屏蔽,提高屏蔽效果。\par 
	 图22\par 
	 
	 
	 \subsubsection{理论分析}
	 两种情况下屏蔽效果是不同的,其原因为:超导环屏蔽原理是迈斯纳效应,当外界磁场发生变化时,超导环会感应超导电流来阻止磁通变化。
	 在场冷条件下,超导在冷却的过程中就获得了一定的磁通Btrap。俘获磁通的增加会导致钉扎势能的降低,其内部感应电流将变小,那么产生的感应磁场也变小,不能抵消外界的变化磁场,所以屏蔽效能减小。
	 被测对象在场冷时的俘获磁通可以下式描述: \par 
	 
	 其中,A为超导环的几何系数;是真空磁导率;Jic表示感应电流,
	 由(3.11)式得到, Jc和r决定超导环的Btrap,故该式对实验结果提供一个理论依据。
	 
	 
	 \subsection{本章小结}
	 本章基于对SQUID工作所需的磁屏蔽技术的要求,研究分析了传统被动磁屏蔽技术、基于高温超导的被动磁屏蔽技术和新型主被动磁屏蔽技术,并开展了相关的实验和理论分析,得出了有助于基于SQUID的磁显微测量微缺陷的磁屏蔽方案。
	 其中,高性能的传统被动屏蔽需要高磁导率材料和多层结构,可以屏蔽恒定与交变磁场,但其设计结构复杂及装置需要极好的封闭性。
	 超导屏蔽根据迈斯纳效应,产生感应电流来抵消变化的磁通量,恒定的外界磁场无法屏蔽掉。
	 而本文提出主动屏蔽与超导屏蔽相结合的方法,既可以实时进行补偿,还可以有效抑制交变磁场的干扰;不仅可以屏蔽轴向磁场,还可以屏蔽横向磁场,大大提高磁场的屏蔽效果,最高屏蔽因数为63dB。在此屏蔽环境下,可以使用高灵敏度SQUID来进行微弱磁场测量,对该装置进一步进行优化,有望可以替代昂贵笨重的磁屏蔽室与屏蔽桶。 
	 
	 
	\newpage
	\section{基于SQUID磁聚焦器研究与设计}
	
	
	\subsection{引言}
	SQUID磁传感器具有高磁场灵敏度,可以测到pT量级的磁场信号,但是用磁测量方法进行样品磁场检测时,为了达到缺陷检测目的,高空间分辨率也是极其重要的问题。磁传感器空间分辨率与灵敏度是相互制约的两个参数,他们与传感器探头的大小有关系,还与传感器到检测样品表面的距离有关系[28]。由于现代微加工技术发展,SQUID传感器可以做得很小,一定程度上解决了传感器探头大小的问题,因此提高空间分辨率的关键为减小磁传感器与样品表面的间距,由高磁导率材料构成的磁聚焦器可有效的提高传感器的空间分辨率,目前已得到广泛应用[48-53]。
	虽然一些研究小组开始利用磁聚焦器进行实验,但是就其性能好坏的依据还没有开展详细的理论分析,如磁聚焦器的退磁因子与哪些因素有关等。本课题将从理论方面进行退磁因子计算,并在室温情况下对其长径比、锥角等进行初步研究,争取最优化设计磁聚焦器与SQUID的耦合,从而提高SQUID磁传感器的空间分辨率。
	
	
	\subsection{磁聚焦器基本理论}
	磁聚焦材料的聚焦能力相当于材料的有效磁导率μ[23],重要的是要注意,由于实际的磁聚焦器件不可能是无限大的,所以与器件的材料磁导率不相同,原因为退磁场的作用影响。材料产生退磁场大小用退磁因子N表示,这些参数之间的关系为:\par
	
	其中,μ代表有效磁导率;μm代表器件的材料磁导率;N表示退磁因子。
	
	
	\subsubsection{退磁场的计算}
	磁聚焦器为圆锥结构,结构参数如下图 23所示。
	图23\par 
	 	对上述磁聚焦器的退磁因子进行分析,根据圆锥与圆柱的结合体的表面磁荷分布,它左右两端的磁荷密度分别为[27, 54]:,。
	 	令轴线上一点P,其到圆形平面的距离为d,几何体的半径为R。由上述可知,圆形面的磁荷面密度,已知点磁荷的磁场强度公式,规定外部施加的磁场方向为正方向,所以圆形面在P点产生的磁场可表示为:\par
	 	
	 	
	 	计算圆锥面的磁荷在轴线上的影响,P点在圆锥面的包围内,与尖端的距离为d0,求出圆锥面上的磁荷在P点的磁场强度为:\par
	 	综上,该几何体轴线上任意一点P处(P在圆锥面的包围内)退磁场的磁场强度H为:\par 
	 	由此可以得到退磁因子ND:\par 
	 	
	 	其中,J为磁极化强度;为材料磁导率;R为该几何体的半径;L为几何体的长度。\par
	\subsubsection{数值分析}
	根据退磁因子最终表达式(4.6),由这个表达式很难直接分析出锥角角度与轴线上退磁因子的关系,因此可以将R和θ以及L的数值关系带入,然后用Maple计算出退磁因子的变化趋势。
	保持R和L不变(即长径比不变),变化θ的大小,如图24,分析退磁因子的变化情况。\par 
	图24\par 
	a.令L=10,R=1保持不变,tanθ=1,则cosθ=。此时退磁因子表达式为:\par 
	
	图25\par 
	以d为横坐标,ND为纵坐标作图,结果如图25所示。由图可知,轴线上中点部分的退磁因子较小,靠近圆形端面的部分退磁因子随着d的增大而快速增大;靠近圆锥面的部分会出现一个峰值,这个峰值的大小远小于靠近圆形端面部分的最大值;并且在0附近的时候,退磁因子的方向与外加磁场方向相同。
	b.令L=10,R=1保持不变,tanθ=3,则cosθ=。此时退磁因子的表达式为:\par 
	
	图26\par 
	以d为横坐标,ND为纵坐标作图,结果如图26所示。由图可知,轴线上中点部分的退磁因子较小,靠近圆形端面的部分退磁因子随着d的增大而快速增大;靠近圆锥面的部分会出现一个峰值,这个峰值的大小远小于靠近圆形端面部分的最大值,且小于a情形中的峰值,并且在0附近的时候,退磁因子的方向与外加磁场方向相同。
	c.令L=10,R=1保持不变,tanθ=5,则cosθ=。此时退磁因子的表达式为:\par 
	
	图27\par 
	以d为横坐标,ND为纵坐标作图,结果如图27所示。由图可知,轴线上中点部分的退磁因子较小,靠近圆形端面的部分退磁因子随着d的增大而快速增大;靠近圆锥面的部分会出现一个峰值,这个峰值的大小远小于靠近圆形端面部分的最大值,且小于b情形中的峰值;并且在0附近的时候,退磁因子的方向与外加磁场方向相同。
	d.令L=10,R=1保持不变,tanθ=10,则cosθ=。此时没有圆柱部分,退磁因子表达式为:
	
	图28\par 
	以d为横坐标,ND为纵坐标作图,结果如图28所示。由图可知轴线上靠近圆形端面的部分退磁因子随着d的增大而快速增大;靠近圆锥面的部分没有出现一个峰值,而是变成了一条近似直线的曲线,随着d的增大退磁因子缓慢增大;在0的附近退磁因子无限接近于0。
	以上4种情形锥角的大小关系a>b>c>d(锥角越来越尖),靠近圆锥面的退磁因子的峰值随着锥角的减小而减小,最终变成一条平滑的曲线。
	由此可知,当合金样品的长径比不变的情况下,样品轴线上靠近中心部分的退磁因子较小,靠近圆形端面的退磁因子较大,靠近圆锥一端的退磁因子会出现一个峰值,峰值出现的点位于圆柱与圆锥的交界处,并且随着锥角变小(圆锥尖端变尖)峰值的大小会越来越小。整体来看,样品轴线上的退磁因子会随着圆锥角度的减小而整体减小。当圆柱部分的长度为0时,轴线上整体的退磁因子减到最小。
	\subsection{有限元仿真分析}
	针对磁聚焦器的多种结构:全圆柱形结构、全圆锥形结构、圆柱与圆锥相结合(锥角为30度)和圆柱与圆锥相结合(锥角为15度),通过有限元分析的方法,分别开展与无磁聚焦器时的仿真对比。由仿真结果可知,磁力线会聚焦通过高磁导率磁聚焦器,从而提高磁场强度,达到聚焦目的。
	进一步通过仿真计算得出磁聚焦器轴线上的二维截线磁场大小及分布,进而判断其聚焦能力,结果如下诸图。不同结构的磁聚焦进行仿真分析,图29(a)为无磁聚焦器时,样品成像图,(b)和(c)图分别为圆柱形和圆锥形结构磁聚焦器,有分析结果可知,全锥形结构磁感应强度是最大的,优于圆柱结构;之后对不同锥角的磁聚焦器进行分析,结果如图30所示,圆锥锥角越小,聚焦能力越好,这也是圆锥结构优于圆柱结构的原因。最后,把所有情况进行分析对比,如图31所示,清楚观察到:加上磁聚焦器后测得磁场会大于无磁聚焦器情况,证明磁聚焦器具有聚焦作用。
	综上所示,仿真结果与上述理论结论相吻合,磁聚焦器可一定程度上解决传感器空间分辨率的问题。
	
	图29\par 
	图30\par 
	图31\par 
	
	
	
	
	\subsection{磁聚焦器磁导率测量}
	\subsubsection{磁导率测量试验装置}
	根据安培电流定理,通电螺线管可以产生均匀的磁场,因此可利用螺线管产生的磁场测量磁性材料的磁化率。图32给出了试验用测试磁化率的螺线管线圈,在该线圈中可以产生均匀的交变磁场。在测试过程中,需要在棒状的被测磁性材料上面也绕上一定匝数如5匝的线圈,如图33所示。其中,绕中心部分的测量方法叫做面磁通门测量方法,测得的退磁因子称为面退磁因子Nf;而全部绕满的测量方法中称体磁通门测量方法,利用测量得的磁通量转化成磁性材料的磁矩,然后通过计算得到的退磁因子称体退磁因子Nm。
	
	图32\par 
	图33\par 
	实际实验设置中,测试系统结构如图34所示,主要包括SIGNAL DISCOVERY的锁相放大器、一个高精度的10Ω的电阻和长为39.2cm、匝数为300的螺线管。其中,锁相放大器一方面为螺线管产生恒定电压Vout,从而确保螺线管内产生均匀磁场;另一方面,锁相放大器基于自身的锁相和放大功能,可检测微弱电信号,实现对被测磁性材料感应信号的采集。锁相放大器所采集的信号包括串接在螺线管上的10Ω电阻两端的Va,和绕在被测材料上线圈两端的Vb。因此,通过Va可以知道磁场中所产生的电流大小,进而可以知道螺线管中产生的磁场大小;而通过Vb可以知道磁性材料上面绕的线圈则作为次级线圈产生的感应电压,从而得到磁性材料在螺线管均匀磁场中所接收的磁通量大小。
	根据法拉第电磁感应定律,依据上述试验过程可以最终推导得出磁性材料的磁化率大小,进而计算得出退磁因子的大小,并迭代至Chen-D-X等文章所列表中,进而完成退磁因子Nf和磁化率大小χ的测量。
	图34\par 
	
	\subsubsection{磁导率测量结果}
	\begin{enumerate}
		\item [(1)]不同长度的磁聚焦器\par 
		用上述测量装置对总长度分别为30mm、50mm和70mm,锥角为60度,直径为6mm的磁聚焦器进行磁导率测量。应用锁相放大器施加不同幅值且频率为155kHz的正弦信号,经数据处理与分析,其结果如下图35所示。
		图35\par 
		从图中可知,磁聚焦器越长,即长径比越大,相对磁导率越大,退磁因子较小。
		\item [(2)]不同锥角的磁聚焦器\par 
		用上述测量装置对总长度为50mm,锥角分别为30、60、120和180度,直径为6mm的磁聚焦器进行磁导率测量。应用锁相放大器施加不同幅值且频率为155kHz的正弦信号,经数据处理与分析,结果如下图36所示。
		图36\par 
		从图中可知,锥角对磁导率有一定影响,但不是特别明显;锥角为60度的磁聚焦器相对磁导率最大。
	\end{enumerate}
	
	\subsection{磁聚焦器应用于磁传感器}
	
	SQUID传感器正常工作时,需要处于低温环境中,一般放置在杜瓦内,而被测样品是室温样品,所以磁传感器与样品之间有一定的距离。为了提高空间分辨率,即减小他们之间的距离,最常用的方法是:将磁聚焦器与传感器相耦合。
	本节将磁聚焦器与三位磁通门计相耦合,结构如图37(a)所示。使用长度为30mm,半径为3mm,锥角为60度的磁聚焦器,三维磁通门计灵敏度为1nT,大小为3232mm,磁聚焦器粘在三维磁通门计的中心位置。磁聚焦器和样品之间的距离为2毫米,样品附在由塑料棒制成的样品架上,可以随着由PC控制的移动平台实现三维运动,运动扫描区域可由PC显示器控制,扫描精度约为0.125。
	扫描显微镜观察的过程为:首先,把耦合磁聚焦器的磁传感器插入到主被动磁屏蔽区域内,放在中心位置;然后,将样品放在样品架上,置于探针头下,设置好初始位置和扫描区域;之后,样品在x和y方向实现扫描;最后采集数据与保存。\par 
	
	图37\par 
	磁场测量样品选用间距为5mm的通电折线,示意图如图37所示。其中,针尖与样品之间距离为2mm,折线施加电流为1A。(c)与(d)图分别显示有无磁聚焦器和磁传感器相耦合所测得电流分布图,从图(c)可以清晰得出四根导线所在位置,而无磁聚焦器测量结果明显观察到两根导线,因此可以得到:磁聚焦器可以明显提高磁传感器的空间分辨率。
	进一步地,对间距为2mm的折线进行磁场测量,样品中共有7根指导线,测试结果如图(e)所示,测试结果可以清楚观察到7根导线电流分布,从而更加证明磁聚焦器在提高磁传感器空间分辨率的作用。
	\subsection{本章小结}
	本章主要从理论上推导出圆锥结构磁聚焦器的退磁因子,并通过实验研究出磁聚焦器聚磁能力与退磁因子相关,退磁因子与材料的有效磁导率、长径比成反比关系。结合上述研究成果,最终得出磁聚焦器的设计应该具有大的长径比,圆锥锥角60度为佳,此时退磁因子小,磁聚焦能力强。
	进一步地,结合后续基于SQUID的磁显微测量微缺陷实验需要,实现了磁聚焦器与高精度磁传感器的耦合,有效地提高了磁传感器的空间分辨率。
		
	
	\newpage
	
	
	
	\section{基于SQUID的磁显微测量微缺陷系统搭建和实验验证与分析}
	
	\subsection{引言}
	前文各章对基于SQUID的磁显微测量微缺陷的相关研究和技术实现为原理样机的搭建奠定了基础,本章首先完成了原理样机的总体方案设计与实现,主要包括SQUID磁传感器部分、磁屏蔽部分、三维微位移运动平台部分和上位机数据分析与处理部分。
	接着,基于原理样机的检测结果,采集数据的后期分析与处理角度入手,研究分析了两种滤波方法,以提高检测效率,并进一步的设计实现了磁场检测结果分布云图的绘制算法,保证检测结果可直观、高效的显示,提高检测系统的可用性。
	最后,以典型的电路结构为检测对象,在此原理样机的基础上开展了一些列的缺陷检测实验。实验结果表明:基于SQUID的磁显微测量微缺陷系统能够以直观、有效的方式获取磁信号,从而反映磁场的缺陷位置与典型类型,实现微缺陷检测的目的,达到了本论文的研究目标。
	\subsection{系统样机总体设计}
	本论文主要研究基于SQUID的磁显微测量微缺陷,基于前文的理论研究和技术实践,搭建了基于SQUID的磁显微测量微缺陷原理样机,结构如图38所示。\par 
	图38\par 
	该原理样机总体设计主要包括:高精度磁传感器、主被动磁屏蔽系统、高精度三维机械运动平台和数据采集与处理分析模块。样机搭建过程需要对每个子模块进行单独调试、运行和实验,然后把所有系统联合调试,最终完成整套系统的搭建,样机实物如图39所示。\par 
	图39\par 
	其中,传感器部分:实验室现有测交变磁场的三维高精度磁传感器,测量量程为10Guass,磁场灵敏度为1nT。另外,超高精度超导量子干涉仪磁传感器是本系统的关键,该传感器系统本征磁通白噪声谱密度小于等于30μ∅0/Hz(1/2),可以测量变化幅度在±300∅0的磁通变化测量,具有磁通噪声低、磁场分辨率高的特点。另外,SQUID的正常工作需要配合技术成熟、价格便宜、安全可靠的液氮作为它的冷却媒质。 
	磁屏蔽部分:无损检测中,被测样品磁场信号为nT量级,地磁场为50000nT,实验需要屏蔽背景磁场来减少噪声和干扰,因此,在超低磁场环境测量中,对环境电磁噪声(如地磁还是城市噪声)进行磁屏蔽是非常重要的。本文磁屏蔽装置为主动磁屏蔽与被动磁屏蔽相结合的方法,详见第三章。
	三维机械运动平台:由运动控制上位机(计算机)、下位机(单片机系统)、步进电机驱动器及步进电机、机械运动装置及试样平台5部分组成。其中,上位机作为控制指令的发送端;下位机在接收到上位机发送来的控制指令后,生成相应的电机运动控制指令;步进电机驱动器直接接收下位机发出的电机控制指令,经过相应的逻辑转换后直接驱动步进电机的动作,进一步由步进电机带动机械滑臂完成最终的空间运动。XYZ三轴组合后的重复定位精度可以达到20µm。
	
	根据基于SQUID的磁显微测量微缺陷的要求,上述原理样机的搭建需要关注一下问题:\par 
	\begin{enumerate}
		\item [(1)]确保 SQUID 在冷却杜瓦真空层降温到工作区间(70K);
		\item [(2)]冷却杜瓦底部靠近样品的部分必须由无磁材料加工制作,以防止其因含有剩磁信号对待测样品磁场信息产生影响;
		\item [(3)]三维移动平台应整体尽量采用无磁材料加工制作,步进电机应在满足基本负荷的范围内,选取最小功率的型号,以减小其噪声对系统的影响;
		\item [(4)]承载样品的样品架,应由绝对无磁材料加工制作而成,且应该保证一定的长度,使得样品远离步进电机磁场干扰的影响;
		\item [(5)]磁屏蔽区域屏蔽磁场尽量均匀、剩磁较少,减少对待测样品的影响。
	\end{enumerate}
	
	
	\subsection{数据采集模块的设计与实现}
	在定量缺陷检测实验中,需要采集数据。在整个检测系统中,数据采集系统完成数据的A/D转换、分析处理和存储等工作。数据采集和存储是由上位机控制界面控制的,操作界面如图40所示,在控制位移平台移动之前,把相应的参数设置好,选上是否采样的选项,然后打开串口,点击开始按钮,在位移平台开始按照运动指令运动的同时,将会进行磁场数据的采样。采样的方式是,采集位移平台所走点阵中的每个点的数据,且每个点上共计采集十个值,以便后期进行数据处理或滤波。在位移平台运动结束后,点击保存数据,选择合适的保存路径,数据会自动保存为一个.txt文档。最后,根据自己的需要处理所保存的数据即可。\par 
	图40\par 
	
	
	\subsection{不同高度z成像}
	扫描SQUID显微镜(SSM)是最敏感的仪器,可以获取样品表面附近的磁场。SQUID扫描显微镜已成为无损检测样品的有力工具,用于微电子、磁通动力学和其他物理领域。为了实现高空间分辨率的SQUID扫描显微镜,SQUID探头应接近样品表面,开发了一种薄蓝宝石窗口的高温SSM,便于在空气中进行样品测量。但是SQUID探头与样品直接的距离取决于蓝宝石窗口的厚度,即不同探头与样品之间的距离,其不同高度处Z方向磁场成像性能尚未进行详细的分析,无法确认SQUID是否能够检测给定样品的实际磁场。
	本节内容,主要研究了不同探测高度处Z分量(BZ)磁场成像性能。用二维分析模型计算了折线的磁场分布,并借助上述原理样机实验验证不同高度处样品的磁场成像结果。
	\newpage
	\section*{结论}
	
	
	
	
	% 参考文献无标号
	\newpage
	\phantomsection\addcontentsline{toc}{section}{参考文献}\tolerance=500 %将参考文献放进目录
	\begin{thebibliography}{10}  
		\bibitem{ref1}COMSOL 中国. \textit{1aaa}\emph{1aaa如何模拟落在水面上的球体?}, 2023[EB/OL]. 知乎-有问题,就会有答案.
		
	\end{thebibliography}
	
%	\newpage
%	\section*{致谢}	
%		\phantomsection\addcontentsline{toc}{section}{致谢}\tolerance=500 %将致谢放进目录
%		感谢大家。
	
	
	
	
	
\end{document}
